%==============================================================================
%== template for LATEX poster =================================================
%==============================================================================
%
%--A0 beamer slide-------------------------------------------------------------
\documentclass[final]{beamer}
\usepackage[orientation=portrait,size=a0,
            scale=1.25         % font scale factor
           ]{beamerposter}
           
\geometry{
  hmargin=2.5cm, % little modification of margins
}

%
\usepackage[utf8]{inputenc}



\linespread{1.15}
%
%==The poster style============================================================
\usetheme{sharelatex}

%==Title, date and authors of the poster=======================================
\title
[CSSS 501 MLE Poster Session, 17 November 2014 -- http://github.com/pschmied/bikecounts-mle] % Conference
{ % Poster title
Weather and Seasonal Factors Affecting Daily Bicycle Counts at Fremont St. Bridge, Seattle
}

\author{ % Authors
Peter Schmiedeskamp\inst{1}, Weiran Zhao\inst{1}
}
\institute
[University of Washington] % General University
{
\inst{1} University of Washington, Seattle
}
\date{\today}



\begin{document}
\begin{frame}[t]
%==============================================================================
\begin{multicols}{3}
%==============================================================================
%==The poster content==========================================================
%==============================================================================

\section{Introduction \& Research Questions}
This project develops a statistical model for predicting the volume of
bicycle traffic per day on the Fremont bridge, in order to better
understand factors believed to influence likelihood of bicycling.
These predictors include weather factors such as precipitation and
temperature, season, day of the week, as well as consideration of a
general trend over time.

\section{Data}
In late 2012, the Seattle Department of Transportation deployed an
automated bicycle detection system that has been collecting bicycle
counts since its installation. These data were downloaded from the
City of Seattle's public data portal, aggregated by day, and then
joined by date to historic weather data queried from Forecast.io. In
this analysis, the collection period starts on October 31, 2012 and
ends October 30, 2014, for a total of 730 daily observations.

These data are summarized graphically in figures \ref{fg:hist} and
\ref{fg:timeseries} below. From the histogram, we see what looks to be
a Poisson distribution with over dispersion. From the timeseries plot,
we see that there is a clear yearly pattern, as well as a weekly
pattern.

\begin{figure}[htbp]
\begin{center}
\includegraphics[width=0.3\textwidth]{../fig/dp1}
\caption{Histogram of bicycle counts}
\label{fg:hist}
\end{center}
\end{figure}

\begin{figure}[htbp]
\begin{center}
\includegraphics[width=0.3\textwidth]{../fig/dp2}
\caption{Counts by day}
\label{fg:timeseries}
\end{center}
\end{figure}


\section{Model Specification}
Because some over dispersion was evident in the counts, we chose to
fit a negative binomial model. This model included the following variables

\begin{table}[htdp]
\begin{center}
\begin{tabular}{ll}
  \hline\hline
  Variable & Description \\
  \hline
  Count* & Number of bicycles / day \\
  Daylight hours & Length of daylight in hours \\
  Holiday & Was the day a holiday? \\
  UW & Was University of Washington in session? \\
  Max temp & Maximum temperature for the day \\
  Max precip & Daily max precip (inches) in any hour \\
  Day number & Sequentially numbered day of study \\
  Sat--Fri & Dummy variables for day of week \\
  \hline\hline
    & \multicolumn{1}{r}{*dependent variable}
\end{tabular}
\end{center}
\label{default}
\end{table}%

\section{Model Fit}

\section{Simulated Counterfactual Results}

\subsection{Temperature}

\begin{figure}[htbp]
\begin{center}
\includegraphics[width=0.3\textwidth]{../fig/m1c1}
\caption{Simulated effect of daily maximum temperature on bicycle counts}
\label{fg:temp}
\end{center}
\end{figure}


\subsection{Precipitation}

\begin{figure}[htbp]
\begin{center}
\includegraphics[width=0.3\textwidth]{../fig/m1c2}
\caption{Simulated effect of daily precipitation (in inches) on bicycle counts}
\label{fg:maxtemp}
\end{center}
\end{figure}

\subsection{Day of the Week}

\begin{figure}[htbp]
\begin{center}
\includegraphics[width=0.3\textwidth]{../fig/m1c3}
\caption{Simulated bicycle counts differences by day of week}
\label{fg:dow}
\end{center}
\end{figure}

\subsection{Seasonality}
In considering how to address seasonality, we were concerned that
Seattle's Pacific Maritime Climate does not reflect a traditional four
season calendar year. We felt that hours of sunshine and whether or
not it was the ``school season'' (i.e. the University of Washington
was in session) were better approximations of season for Seattle.

\begin{figure}[htbp]
\begin{center}
\includegraphics[width=0.3\textwidth]{../fig/m1c4}
\caption{Effect of season (measured continuously as number of daylight hours)}
\label{default}
\end{center}
\end{figure}


\section{Conclusions}
%==============================================================================
%==End of content==============================================================
%==============================================================================

%--References------------------------------------------------------------------

\subsection{References}

\begin{thebibliography}{99}

\bibitem{ref1} J.~Doe, Article name, \textit{Phys. Rev. Lett.}

\bibitem{ref2} J.~Doe, J. Smith, Other article name, \textit{Phys. Rev. Lett.}

\bibitem{web} \url{http://www.google.pl}

\end{thebibliography}
%--End of references-----------------------------------------------------------

\end{multicols}

%==============================================================================
\end{frame}
\end{document}
