% !TeX TS-program = pdflatexmk
% !BIB TS-program = biber

\documentclass[12pt,letterpaper,article]{memoir} % for a short document
\chapterstyle{article}

% Bibliography
\usepackage[authordate,strict,bibencoding=inputenc,doi=false,isbn=false,annotation=true]{biblatex-chicago}

% Graphics
\usepackage{graphicx}
\DeclareGraphicsExtensions{.pdf,.png,.jpg}
\addbibresource{../bibliography/bikecounts.bib}

\title{Estimating Bicycle Counts from Seasonal and Weather Factors}
\author{Peter Schmiedeskamp, Weiran Zhao}
%\date{8 October 2014} % Delete this line to display the current date

%%% BEGIN DOCUMENT
\begin{document}

\maketitle

\section*{Introduction}
%
% 1. Explain what we set out to do (i.e. understand influence of
% seasonal and weather factors on bicycling) 2. explain relevance to
% planners (e.g. understanding how much seasonal variability to expect
% for service provision, understand methodologically how to use data
% collected from counters) 3. ??? other things I'm forgetting right
% now.
% 

\section*{Literature review}
% 
% Identify where people have talked about weather and seasonal factors
% influencing counts previously. Describe their methods together with any
% limitations, as well as their substantive findings---especially what
% variables they considered. Normally should conduct this first, but
% our driver here is a class assignment! We should fill this in and
% adjust our analysis accordingly before attempting to go to press.
% 

\section*{Methodology}
In order to discern the relationship between bicycle counts and
several identified weather, seasonal, and temporal factors, we
developed a statistical model that attempts to predict daily bicycle
counts from these other factors. This section describes the methods
and procedures we used to collect and process the estimation dataset,
the rationale for our selection of variables, our chosen model type,
and our model estimation procedures.
\subsection*{Data collection, processing, description}
%
% Data collection: study site, descriptives of sample, collection from
% forecast.io and seattle.gov, not in that order
% 

The data that we attempt to explain were collected at Seattle's
Fremont Bridge and cover a period of two years spanning from October
31, 2012 to October 30, 2014. The Fremont Bridge captures a
substantial number of bicyclists due to its status as one of only five
facilities that carry bicyclists across the canal that separates the
northern and southern halves of Seattle. Bicycle counts are collected
at this location continuously by the City of Seattle using an
in-sidewalk counter manufactured by EcoCounter. When a bicycle passes
over an induction loop embedded in the sidewalk on either side of the
Fremont Bridge, the counter registers the bicycle. Bicyclists may
legally choose to ride in the roadway instead of the sidewalk, and
would not be detected by the counter. However, we believe these
crossings are rare at this location due to the design of the facility,
which directs bicyclists to enter the sidewalk, and from our own
experience riding and observing other riders. Counts are aggregated
into 15 minute intervals, and are made available to the public via the
City of Seattle's data
portal \parencite{City-of-Seattle:aa,City-of-Seattle:ab}.

Weather data are aggregated and made available through Forecast.io's
web services API \parencite{The-Dark-Sky-Company:aa}. Historical daily
summaries are available for a range of weather variables including
several specifically important to our model such as precipitation,
daily minimum and maximum temperatures, sunrise, and sunset.

We downloaded and processed these data programmatically using the R
programming language along with several add-on
packages \parencite{Grolemund:2011aa,Wickham:2011aa,Couture-Beil:2014aa,Lang:2014aa,R-Core-Team:2014aa}.
Bicycle counts were aggregated by day, and then joined to weather data
by date. In addition to the variables collected from these two
sources, we were also interested in controlling for holidays and
whether or not the nearby University of Washington was in session.
These data were collected and coded manually from the University of
Washington's historic academic calendars.

Figures \ref{fg:hist} and \ref{fg:timeseries} provide visual summaries
of the processed counts data.

\begin{figure}[h!]
  \centering
  \includegraphics[width=0.9\textwidth]{../fig/dp1}
  \caption{Histogram showing frequency of observed bicycles}
  \label{fg:hist}
\end{figure}

\begin{figure}[h!]
  \centering
  \includegraphics[width=0.9\textwidth]{../fig/dp2}
  \caption{Timeseries plot of bicycle counts}
  \label{fg:timeseries}
\end{figure}


\subsection*{Variable selection}
% 
% Data analysis: specification of model, choice of model (nb v pois),
% description of counterfactual simulation using the estimated model,
% software used.
% 

Out of the set independent variables retrieved from Forecast.io and
the University of Washington, we selected a subset that we felt best
reflected our specific research questions. The final set of variables
included in our model include Table \ref{tb:variables} summarizes the
variables used in our model.

\begin{centering}
\begin{table}[h]
\caption{Variables included in model specification}
\begin{tabular}{ll}
\toprule
Variable & Description \\
\midrule
Count* & Number of bicycles per day \\
Daylight hours & Length of daylight in hours \\
UW in session & Was the University of Washington in session? \\
Holiday & Was the day a holiday as recognized by UW? \\
Max temp & Maximum temperature for the day \\
Max precip & Daily maximum inches of precipitation in any hour \\
Sat--Fri & Day of the week dummy variable (relative to Sunday) \\
Day number & Sequentially numbered day of study \\
\bottomrule
\multicolumn{2}{r}{* Dependent variable}
\end{tabular}

\label{tb:variables}
\end{table}
\end{centering}

Daylight hours (defined as $sunrise - sunset$) and the University of
Washington in-session status were selected to represent seasonality.
Daylight hours was chosen instead of a calendar-based categorization
of season in part because Seattle's Pacific Maritime Climate differs
substantially from traditional notions of Spring, Summer, Autumn, and
Winter. Daylight hours also is measured as a continuous value at a
finer temporal resolution of one day. Finally, daylight hours adjusts
according to latitude, which may make this model estimation procedure
and specification more transferable to other sites in the future.

We deemed the University of Washington variable important in part
because of the Fremont Bridge's proximity and connection via the Burke
Gilman Trail to the University of Washington. We also felt that this
variable was a suitable proxy for the ``school season,'' which more
broadly captures whether or not local schools are in session. The
academic calendars of the various local schools do not align
perfectly, however they still overlap substantially with the
University of Washington, which is itself the largest educational
institution in the region.

Inclusion of the holiday variable was an attempt to account for
some low outlier counts. Upon inspection of the dataset, Christmas and
Thanksgiving in particular had very low counts of bicycles relative to
the days preceding and following. Relatedly, but not accounted for by
any variable in our model, are some of the high outlier counts. Upon
inspection, some of the highest counts were observed on National Bike
to Work Day and on the day of the Fremont Solstice Parade, which
typically draws large numbers of bicyclists as participants and
spectators. The omission of such a variable is justified based on the
few occurrences of high outlier counts, and our desire for this model
to only include variables that could be collected or straightforwardly
adapted to other locations.

Daily maximum temperature, measured in Fahrenheit, was chosen to
represent temperature (rather than, for example, substituting or
adding daily minimum temperature) in part to retain simplicity in the
model, in part because there is relatively little daily temperature
variation in Seattle due to the moderating effect of large water
bodies, and in part because maximum temperature better reflects the
conditions during daylight hours when most bicycle trips would occur.
This simplification may not be warranted for other locations that
experience greater temperature variation than Seattle.

Maximum precipitation, which measures the maximum inches of
precipitation that occurred in any hour throughout the day, was chosen
rather than average precipitation based on the notion that bicyclists
might make travel decisions based on a likely worst case scenario.
This assumption is slightly more problematic than our assumptions
about temperature, in that we do expect bicyclists to be at least
somewhat sensitive to average conditions or conditions observed at
their time of departure. As in the case of temperature, this
simplification would be less justifiable in locations that experience
greater daily variation in precipitation or in locations that have a
predictable pattern of precipitation during certain hours.

Day of the week was added due to its presence in the literature, as
well as an apparent weekly pattern revealed visually by zooming into
the timeseries plot. These data were coded as a set of Boolean dummy
variables, excluding Sunday as the reference category.

The final variable, the day number, was included so that we could test
for a linear trend in bicycling volumes. We created this variable by
sequentially numbering (1--720) the observed counts by day during the study period.

\subsection*{Model estimation}
% Model choice
A natural model choice for count data is the Poisson model, however we
believe that our count data may be overdispersed. Overdispersion, or
contagion between events, violates the mean-variance equivalence
assumption of the Poisson model. While overdispersion would not impact
our parameter estimates, it would result in overly optimistic margins
of error. In order to account for and estimate the amount of
overdispersion present in our data, we chose a negative binomial model
type. Because there were no days observed with zero bicycle counts, we
did not need to resort to a zero-inflated model as is often necessary
when dealing with count data.

% Estimation method
We fit the model in R using the \texttt{glm.nb} function from the MASS
package \parencite{Venables:2002aa}. For comparison, we also estimated a
Poisson model with an analogous specification in R using the
\texttt{glm} function. A much lower AIC and BIC in the case of the
negative binomial model confirmed that it was a better fit than the
Poisson. We then tested for overdispersion using the \texttt{odTest} function
from the Pscl package \parencite{Jackman:2014aa}. The highly significant
chi-square test statistic provided strong evidence that overdispersion
was present in the data, further confirming the choice of a negative
binomial.

In addition to testing the model fit relative to its Poisson
analog, we visually assessed the fit of our negative binomial model by
plotting actual versus predicted values as shown in figure
\ref{fg:avp}. This fit appears to be generally good, though very high
count days are predicted less accurately.

\begin{figure}[h!]
  \centering
  \includegraphics[width=0.7\textwidth]{../fig/avpp2-sm}
  \caption{Actual versus predicted values as fit by negative binomial model}
  \label{fg:avp}
\end{figure}

% Model interpretation
In order to provide results that are more readily interpretable by
non-statisticians, we used counterfactual simulation to isolate
individual terms from the model that correspond to our research
questions. In so doing, we generated various quantities of interest
including point estimates and confidence intervals, and then plotted
them for visual inspection. Counterfactual simulations were performed
with a modified version of the Simcf R package, and visualized with
ggplot2 \parencite{Adolph:2014aa,Schmiedeskamp:aa,Wickham:2009aa}.
Results of these simulations are presented in the following section.

\section*{Results}
% 
% One subsection for each research question
% 
This section presents the results from the statistical model and
accompanying counterfactual simulations as described in the preceding
section. Each of the main research questions of seasonality, weather,
and general trend are addressed here. In addition, additional added
control variables such as day of the week are presented.

Without exception, each of the coefficients in our model was
statistically significant at the $p < 0.05$. Further, with the
exception of the Saturday coefficient, all coefficients were
significant at the $p < 0.001$ level. The remainder of this section
focuses on presenting the substantial effect of each variable.

\subsection*{Seasonality}
As discussed previously, we considered two variables to address the
question of seasonality: the first being the number of daylight hours,
and the second being whether or not the University of Washington
(proxying more generally for other educational institutions) is in
sesssion.

As seen in figure \ref{fg:seasonality}, we see a substantial increase
in bicycle volumes when the University of Washington is in session.
With all other factors held constant we see that, on days when the
University is in session, there is an average of approximately 367
additional bicycles observed. Similarly, we see an almost linear
increase in bicycles with increased day length.

\begin{figure}[h!]
  \centering
  \includegraphics[width=.8\textwidth]{../fig/m1c4}
  \caption{Effect of daylight hours and University of Washington
    in-session status on bicycle counts, with shaded 95\% confidence
    regions.}
  \label{fg:seasonality}
\end{figure}


\subsection*{Weather}

\begin{figure}[h!]
  \centering
  \includegraphics[width=.8\textwidth]{../fig/m1c2}
  \caption{Effect of precipitation on counts, with shaded 95\%
    confidence region.}
  \label{fg:precipitation}
\end{figure}

\begin{figure}[h!]
  \centering
  \includegraphics[width=.8\textwidth]{../fig/m1c1}
  \caption{Effect of temperature on counts, with shaded 95\%
    confidence region.}
  \label{fg:temperature}
\end{figure}



\subsection*{General trend in bicycle counts}

\begin{figure}[h!]
  \centering
  \includegraphics[width=.8\textwidth]{../fig/m1c5}
  \caption{General trend in bicycling counts, all other factors held
    constant, with 95\% confidence region.}
  \label{fg:trend}
\end{figure}



\subsection*{Day of week variation}

\begin{figure}[h!]
  \centering
  \includegraphics[width=.8\textwidth]{../fig/m1c3}
  \caption{Variation in counts throughout the week, with 95\% confidence bars.}
  \label{fg:dayofweek}
\end{figure}

\clearpage
\section*{Discussion}


\section*{Conclusion}

\printbibliography
\end{document}