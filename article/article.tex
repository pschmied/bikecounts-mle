% !TeX TS-program = pdflatexmk
% !BIB TS-program = biber

\documentclass[12pt,letterpaper,article,twocolumn]{memoir}

% Layout options
\chapterstyle{article}
\setulmarginsandblock{.5in}{.75in}{*}
\setlrmarginsandblock{.5in}{*}{*}
\checkandfixthelayout
\raggedbottom

% Bibliography
\usepackage[authordate,strict,bibencoding=inputenc,doi=false,isbn=false,annotation=true]{biblatex-chicago}

% Graphics
\usepackage{graphicx}
\DeclareGraphicsExtensions{.pdf,.png,.jpg}
\addbibresource{../bibliography/bikecounts.bib}

% Float barriers
\usepackage[section]{placeins}

\title{Estimating Daily Bicycle Counts in Seattle from Seasonal and Weather Factors}
\author{Peter Schmiedeskamp, Weiran Zhao}
%\date{8 October 2014} % Delete this line to display the current date

%%% BEGIN DOCUMENT
\begin{document}

\maketitle

\section*{Introduction}
%
% 1. Explain what we set out to do (i.e. understand influence of
% seasonal and weather factors on bicycling) 2. explain relevance to
% planners (e.g. understanding how much seasonal variability to expect
% for service provision, understand methodologically how to use data
% collected from counters) 3. ??? other things I'm forgetting right
% now.
% 

Bicycling offers a variety of social benefits that include the health
of those choosing to bicycle regularly, as well as more broadly a
society-wide increase in mobility as people shift from congestion
causing modes to bicycling. As such, bicycling is being promoted by
policy makers and urban planners as an increasingly important element
of urban transportation. In order to develop policy and
infrastructural improvements to induce more bicycling, a robust
understanding of factors related to bicycling is required.

This article contributes to that understanding by focusing on the
substantive effects of weather and seasonality on the number of
bicyclists observed on a given day. In order to address these two
primary research questions, we develop a statistical model estimated
using two years of automated bicycle counts collected on Seattle's
Fremont Street bridge, as well as historic weather and season data
collected from other sources.

Using this model, we were able to identify and quantify substantial
effects on bicycle counts associated with temperature, precipitation,
hours of daylight, and school session status. Additionally, we
estimated changes in bicycle numbers associated with the day of the
week and holidays, as well as an overall trend toward increased
bicycling.

\section*{Literature review}
% 
% Identify where people have talked about weather and seasonal factors
% influencing counts previously. Describe their methods together with any
% limitations, as well as their substantive findings---especially what
% variables they considered. Normally should conduct this first, but
% our driver here is a class assignment! We should fill this in and
% adjust our analysis accordingly before attempting to go to press.
% 

Bicycling volumn in cities iuseful for practitioners and researchers
to understand safety, travel behavior, and development impacts.
Therefore it is of great interest to investigate the relationship
between bicycle volume and various factors. The goal is to build a
predictive model based on this relationship, so that one can better
estimate and forecast bicycle travel, help bicycle facility planning,
etc. There have been a lot of papers in the last decade related to
this topic (see, for example,
\cite{Griswold:2011aa,Fields:2012aa,Niemeier:1996aa,Nosal:2014aa}, and
reference therein). Regarding data source, survey/census data are more
used to explain influencing factors such as demographic and
socioeconomic factors to mode choice \cite{Parkin:2008aa,Helbich:2014aa} while
self-collected data(automatic or manual) are used in more studies in
recently to track and analyze the count in a long period of
time\cite{Griswold:2011aa,Nosal:2014aa}. Despite a great amount of research,
the methodology is more or less the same: 1) First propose a set of
explanatory variables; 2) Then fit a proper regression model and
justify the result.

According to a recent report by \cite{Bassok:2011aa}, there are eleven
primary indicators, which are time of day \cite{Schwartz:1999aa},
season \cite{Niemeier:1996aa}, population and employment densities
\cite{McCahil:2008aa,Pinjari:2009aa}, mix of use
\cite{Pinjari:2009aa}, bicycle facility type \cite{Hunt:2007aa},
traffic volume \cite{McDonald:2007aa}, rain and temperature
\cite{Niemeier:1996aa,Parkin:2008aa}, income \cite{Turner:1998aa}, and
age \cite{Hunt:2007aa}. This section outlines the key points made in
the literature that are relevant to some of most important variables.
Research has found the variability for counts has a positive
association with high temperature and low precipitation
\cite{Niemeier:1996aa,Parkin:2008aa}. Meanwhile, as suggested by
\cite{Lewin:2011aa} and \cite{Thomas:2009aa}, the effects of
precipitation and temperature on bicycle volumes are nonlinear. For
example, bicycle traffic can decrease in both very cold and very hot
weather as noted in \cite{Richardson:2000aa}. Apart from the usual
temperature and rain variables, \cite{Miranda-Moreno:2011aa} finds
humidity and additional precipitation variables including the presence
of rain in the morning and/or during the previous three hours to be
significant too. Other comparative studies are also available where
bicycle counts are conducted in different cities and different
sensitives to weather are examined \cite{Rose:2011aa}. As for
longitudinal study, \cite{Niemeier:1996aa} finds increased variability
for counts conducted in the later months of the year.
\cite{Jones:2010aa} conclude that morning peak hours from 6 AM to 9 PM
accounts for a consistent 95\% of the total bicycle volumes by hourly
count data. The simple linear regression model has been used in the
many general applications \cite{Jones:2008aa,Jones:2010aa}. Few
exceptions include \cite{Miranda-Moreno:2011aa} which develops a count
model and \cite{Thomas:2009aa} which develops a time-series model.
\cite{Niemeier:1996aa} also uses a Poisson model to statistically
confirm that many of the factors thought to influence cyclists. The
work by \cite{Gallop:2012aa} adopts a similar time-series approach
while incorporating an autoregressive integrated moving average
(ARIMA) analysis.

After reviewing these literature, we found that the bicycle count data
is need to be further analysis with a long-term data and a better
model. How seasonal factors influence bicycle flow needs to be
examined in data that last more than a year. Plus, among the past
researches, almost no literature discuss about the goodness of fit of
their modeling, a model that can better describe and forecast the
bicycle count in longitudinal form is necessary to developed. Models
for count data with better estimation methods are promising.




\section*{Methodology}
In order to discern the relationship between bicycle counts and
several identified weather, seasonal, and temporal factors, we
developed a statistical model that attempts to predict daily bicycle
counts from these other factors. This section describes the methods
and procedures we used to collect and process the estimation dataset,
the rationale for our selection of variables, our chosen model type,
and our model estimation procedures.

\subsection*{Data collection, processing, description}
%
% Data collection: study site, descriptives of sample, collection from
% forecast.io and seattle.gov, not in that order
% 

The data that we attempt to explain were collected at Seattle's
Fremont Bridge and cover a period of two years spanning from October
31, 2012 to October 30, 2014. The Fremont Bridge captures a
substantial number of bicyclists due to its status as one of only five
facilities that carry bicyclists across the canal that separates the
northern and southern halves of Seattle. Bicycle counts are collected
at this location continuously by the City of Seattle using an
in-sidewalk counter manufactured by EcoCounter. When a bicycle passes
over an induction loop embedded in the sidewalk on either side of the
Fremont Bridge, the counter registers the bicycle. Bicyclists may
legally choose to ride in the roadway instead of the sidewalk, and
would not be detected by the counter. However, we believe these
crossings are rare at this location due to the design of the facility,
which directs bicyclists to enter the sidewalk, and from our own
experience riding and observing other riders. Counts are aggregated
into 15 minute intervals, and are made available to the public via the
City of Seattle's data
portal \parencite{City-of-Seattle:aa,City-of-Seattle:ab}.

Weather data are aggregated and made available through Forecast.io's
web services API \parencite{The-Dark-Sky-Company:aa}. Historical daily
summaries are available for a range of weather variables including
several specifically important to our model such as precipitation,
daily minimum and maximum temperatures, sunrise, and sunset.

We downloaded and processed these data programmatically using the R
programming language along with several add-on
packages \parencite{Grolemund:2011aa,Wickham:2011aa,Couture-Beil:2014aa,Lang:2014aa,R-Core-Team:2014aa}.
Bicycle counts were aggregated by day, and then joined to weather data
by date. In addition to the variables collected from these two
sources, we were also interested in controlling for holidays and
whether or not the nearby University of Washington was in session.
These data were collected and coded manually from the University of
Washington's historic academic calendars.

Figures \ref{fg:hist} and \ref{fg:timeseries} provide visual summaries
of the processed counts data.

\begin{figure}[h!]
  \centering
  \includegraphics[width=0.5\textwidth]{../fig/dp1}
  \caption{Histogram showing frequency of observed bicycles}
  \label{fg:hist}
\end{figure}

\begin{figure}[h!]
  \centering
  \includegraphics[width=0.5\textwidth]{../fig/dp2}
  \caption{Timeseries plot of bicycle counts}
  \label{fg:timeseries}
\end{figure}


\subsection*{Variable selection}
% 
% Data analysis: specification of model, choice of model (nb v pois),
% description of counterfactual simulation using the estimated model,
% software used.
% 

Out of the set independent variables retrieved from Forecast.io and
the University of Washington, we selected a subset that we felt best
reflected our specific research questions. The final set of variables
included in our model include Table \ref{tb:variables} summarizes the
variables used in our model.

\begin{centering}
\begin{table}[h]
\begin{scriptsize}
\caption{Variables included in model specification}
\begin{tabularx}{0.5\textwidth}{>{\raggedright\arraybackslash}p{1.6cm}>{\raggedright\arraybackslash}X}
\toprule
Variable & Description \\
\midrule
Count* & Number of bicycles per day \\
Daylight  & Length of daylight in hours \\
UW & Was the University of Washington in session? \\
Holiday & Was the day a holiday as recognized by UW? \\
Max temp & Maximum temperature for the day \\
Max precip & Daily maximum inches of precipitation in any hour \\
Sat--Fri & Day of the week dummy variable (relative to Sunday) \\
Day \# & Sequentially numbered day of study \\
\bottomrule
\multicolumn{2}{r}{* Dependent variable}
\end{tabularx}

\label{tb:variables}
\end{scriptsize}
\end{table}
\end{centering}

Daylight hours (defined as $sunset - sunrise$) and the University of
Washington in-session status were selected to represent seasonality.
Daylight hours was chosen instead of a calendar-based categorization
of season in part because Seattle's Pacific Maritime Climate differs
substantially from traditional notions of Spring, Summer, Autumn, and
Winter. Daylight hours also is measured as a continuous value at a
finer temporal resolution of one day. Finally, daylight hours adjusts
according to latitude, which may make this model estimation procedure
and specification more transferable to other sites in the future.

We deemed the University of Washington variable important in part
because of the Fremont Bridge's proximity and connection via the Burke
Gilman Trail to the University of Washington. We also felt that this
variable was a suitable proxy for the ``school season,'' which more
broadly captures whether or not local schools are in session. The
academic calendars of the various local schools do not align
perfectly, however they still overlap substantially with the
University of Washington, which is itself the largest educational
institution in the region.

Inclusion of the holiday variable was an attempt to account for
some low outlier counts. Upon inspection of the dataset, Christmas and
Thanksgiving in particular had very low counts of bicycles relative to
the days preceding and following. Relatedly, but not accounted for by
any variable in our model, are some of the high outlier counts. Upon
inspection, some of the highest counts were observed on National Bike
to Work Day and on the day of the Fremont Solstice Parade, which
typically draws large numbers of bicyclists as participants and
spectators. The omission of such a variable is justified based on the
few occurrences of high outlier counts, and our desire for this model
to only include variables that could be collected or straightforwardly
adapted to other locations.

Daily maximum temperature, measured in Fahrenheit, was chosen to
represent temperature (rather than, for example, substituting or
adding daily minimum temperature) in part to retain simplicity in the
model, in part because there is relatively little daily temperature
variation in Seattle due to the moderating effect of large water
bodies, and in part because maximum temperature better reflects the
conditions during daylight hours when most bicycle trips would occur.
This simplification may not be warranted for other locations that
experience greater temperature variation than Seattle.

Maximum precipitation, which measures the maximum inches of
precipitation that occurred in any hour throughout the day, was chosen
rather than average precipitation based on the notion that bicyclists
might make travel decisions based on a likely worst case scenario.
This assumption is slightly more problematic than our assumptions
about temperature, in that we do expect bicyclists to be at least
somewhat sensitive to average conditions or conditions observed at
their time of departure. As in the case of temperature, this
simplification would be less justifiable in locations that experience
greater daily variation in precipitation or in locations that have a
predictable pattern of precipitation during certain hours.

Day of the week was added due to its presence in the literature, as
well as an apparent weekly pattern revealed visually by zooming into
the timeseries plot. These data were coded as a set of Boolean dummy
variables, excluding Sunday as the reference category.

The final variable, the day number, was included so that we could test
for a linear trend in bicycling volumes. We created this variable by
sequentially numbering (1--720) the observed counts by day during the
study period.

\subsection*{Model estimation and goodness of fit}
% Model choice
A natural model choice for count data is the Poisson model, however we
believed that our count data were overdispersed. Overdispersion, or
contagion between events, violates the mean-variance equivalence
assumption of the Poisson model. While overdispersion would not impact
our parameter estimates, it would result in overly optimistic margins
of error. In order to account for and estimate the amount of
overdispersion present in our data, we chose a negative binomial model
type. Because there were no days observed with zero bicycle counts, we
did not need to resort to zero-inflated models as is often necessary
when dealing with count data.

% Estimation method
We fit the model in R using the \texttt{glm.nb} function from the MASS
package \parencite{Venables:2002aa}. For comparison, we also estimated a
Poisson model with an analogous specification in R using the
\texttt{glm} function. A much lower AIC and BIC in the case of the
negative binomial model confirmed that it was a better fit than the
Poisson. We then tested for overdispersion using the \texttt{odTest} function
from the Pscl package \parencite{Jackman:2014aa}. The highly significant
chi-square test statistic provided strong evidence that overdispersion
was present in the data, further confirming the choice of a negative
binomial.

In addition to testing the model fit relative to its Poisson
analog, we visually assessed the fit of our negative binomial model by
plotting actual versus predicted values as shown in figure
\ref{fg:avp}. This fit appears to be generally good, though very high
count days are predicted somewhat less accurately.

\begin{figure}[h!]
  \centering
  \includegraphics[width=0.5\textwidth]{../fig/avpp2-sm}
  \caption{Actual versus predicted values as fit by negative binomial model}
  \label{fg:avp}
\end{figure}

% Model interpretation
In order to provide results that are more readily interpretable by
non-statisticians, we used counterfactual simulation to isolate
individual terms from the model that correspond to our research
questions. In so doing, we generated various quantities of interest
including point estimates and confidence intervals, and then plotted
them for visual inspection. Counterfactual simulations were performed
with a modified version of the Simcf R package, and visualized with
ggplot2 \parencite{Adolph:2014aa,Schmiedeskamp:aa,Wickham:2009aa}.
Results of these simulations are presented in the following section.

\section*{Results}
% 
% One subsection for each research question
% 
This section presents the results from the statistical model and
accompanying counterfactual simulations as described in the preceding
section. Each of the main research questions of seasonality, weather,
and general trend are addressed here. In addition, additional added
control variables such as day of the week, holidays, and linear trend
are presented.

Without exception, each of the coefficients in our model was
statistically significant at the $p < 0.05$. Further, with the
exception of the Saturday coefficient, all coefficients were
significant at the $p < 0.001$ level. The remainder of this section
focuses on presenting the substantive effect of each variable.

\subsection*{Seasonality}
As discussed previously, we considered two variables to address the
question of seasonality: the first being the number of daylight hours,
and the second being whether or not the University of Washington
(proxying more generally for other educational institutions) was in
session.

Figure \ref{fg:seasonality} shows that we see a substantial increase
in bicycle volumes when the University of Washington is in session.
With all other factors held constant we see that, on days when the
university is in session, there is an average of approximately 367
additional bicycles observed. Similarly, we see a roughly linear
increase in bicycles with increased day length.

\begin{figure}[htbp!]
  \centering
  \includegraphics[width=.5\textwidth]{../fig/m1c4}
  \caption{Effect of daylight hours and University of Washington
    in-session status on bicycle counts, with shaded 95\% confidence
    regions.}
  \label{fg:seasonality}
\end{figure}

\FloatBarrier
\subsection*{Weather}
As in the case of seasonality, our model represents weather using two
variables, in this case precipitation (measured as the maximum amount
of precipitation falling in any hour of that day), and maximum
temperature (measured daily, in Fahrenheit).

The effect of precipitation is shown in figure \ref{fg:precipitation}.
From this, we can see a clear inverse relationship between
precipitation and bicycle counts. The rate of decrease in bicycles
appears to begin somewhat steeply, and then begins to slow slightly at
higher amounts of precipitation. This suggests that people are
generally more sensitive to the presence of precipitation than the
intensity.

Temperature, in contrast to precipitation, has a clearly positive
association with increased numbers of bicyclists. We were somewhat
surprised to not see a leveling off in counts at very high
temperatures---likely due to Seattle's moderate summer climate, and the
lack of extremely warm days in our dataset.

\begin{figure}[ht!]
  \centering
  \includegraphics[width=.5\textwidth]{../fig/m1c2}
  \caption{Effect of precipitation on counts, with shaded 95\%
    confidence region.}
  \label{fg:precipitation}
\end{figure}

\begin{figure}[ht!]
  \centering
  \includegraphics[width=.5\textwidth]{../fig/m1c1}
  \caption{Effect of temperature on counts, with shaded 95\%
    confidence region.}
  \label{fg:temperature}
\end{figure}


\FloatBarrier
\subsection*{General trend in bicycle counts}
Our results (shown in figure \ref{fg:trend}) confirm the presence of a
general trend toward increased numbers of bicycles at this location.
With all else held constant, we would expect to see roughly 300 more
bicycles on days at the end of our study period than at the beginning.
This is consistent with figures reported elsewhere that suggest that
bicycling is increasing in a number of cities including
Seattle \parencite{League-of-American-Bicyclists:aa}.

\begin{figure}[ht!]
  \centering
  \includegraphics[width=.5\textwidth]{../fig/m1c5}
  \caption{General trend in bicycling counts, all other factors held
    constant, with 95\% confidence region.}
  \label{fg:trend}
\end{figure}


\FloatBarrier
\subsection*{Day of week variation}
As discussed previously, day of the week was included due to the
weekly variation in bicycle counts apparent in timeseries plots.
Figure \ref{fg:dayofweek} shows several interesting aspects of these
results. First, we see much higher numbers of bicyclists on weekdays
than on weekends. This strongly suggests that the majority of the
bicycle traffic at this location is for commuter purposes.

Comparing between weekday results, we see that most days are roughly
the same, with the some drop-off toward Thursday and Friday. This
decrease toward the end of the work week might be attributable to
individuals either working non-traditional schedules or perhaps people
adjusting their travel mode choice in order to accommodate social
engagements.

\begin{figure}[ht!]
  \centering
  \includegraphics[width=.5\textwidth]{../fig/m1c3}
  \caption{Variation in counts throughout the week, with 95\% confidence bars.}
  \label{fg:dayofweek}
\end{figure}

\FloatBarrier
\section*{Discussion}
This research set out to better understand the relationship between
bicycle counts and weather and season. However, weather and season are
themselves comprised by a number of constituent elements. We believe
this research has chosen defensible representations of weather and
season. However, we readily concede that there exist other reasonable
representations of season and weather.

In particular, others might be more interested in building a model
that places more emphasis on predictive power. Those individuals might
choose to add a number of additional variables or consider more
complex interactions between terms. By contrast, in this model
formulation, we present a relatively simple model with comparatively
fewer terms. This was due to our interest in a conceptual
understanding of the influence of weather and season.

Even within our model, as noted in the methodology section, we
considered alternative specifications, especially those containing
different measures of temperature and precipitation. As an example, in
choosing to represent temperature as the daily maximum, we make some
assumptions about people and their mode choice decision making
processes. People cannot know the daily maximum temperature in advance
of their morning commute, however the measured daily maximum
temperature is likely to be similar to temperatures reported in
temperature forecasts. A superior measure might be a consensus of
forecasts available in the morning of each day. Similarly, if
Seattle's daily high and daily low temperatures did not have such low
variation, a daily maximum temperature might be less important to a
winter-time bicyclist than daily minimum.

One variable missing from our analysis that we would have liked to
include is cloud cover.\footnote{We gratefully acknowledge Christopher
  Adolph for this suggestion.} This variable would have allowed us to
account for the possibility of brilliant, sunny, but cold winter days,
as well as warm, dry, but gloomy summer days. Unfortunately, the
dataset downloaded from Forecast.io included a great deal of missing
values for this variable, and we were unable to identify another
suitable source for this data in time for this article.

Another limitation of this research comes from our sample of bicycle
counts. While we do have continuous counts spanning a full two years,
these counts were taken at just one point location. We believe the
Fremont bridge is somewhat representative of a high-volume bicycle
facility in Seattle, however we might, for example, see a decreased
effect size for University of Washington session status if we looked
at counts further from the university. The City of Seattle has begun
collecting and releasing data from other counters, which could be
included in a future version of this analysis.

Finally, as discussed in the methodology section, we believe the
choice of the negative binomial model to be a better choice than the
Poisson, which had been used in some previous studies. A limitation in
this model, however, is in not accounting for timeseries
autocorrelation. We think it highly likely that the decision to
bicycle on one day might influence the decision to bicycle on the
next. A future direction would be to consider models, such as ARIMA,
that more explicitly account for time.

\section*{Conclusion}
This research set out to help clarify the relationship between weather
and seasonal factors and bicycle counts in Seattle. In order to
achieve this, we developed a negative binomial model to predict
bicycle counts based on temperature, precipitation, day length,
university in-session status, day of the week, and a general linear
time trend.

For each term included in our model, we found statistically
significant effects. More importantly, through the use of
counterfactual simulation, we estimated what we deem to be substantial
effect sizes associated with each predictor variable.

This article contributes to the existing literature by demonstrating
the use of a negative binomial model for bicycle counts. In addition,
the results presented here were generated from data collected over a
long study period of two years.

While control of the weather and seasons are admittedly beyond the
scope of policy makers, this research does suggest that planners and
policy makers may want to develop strategies that help mitigate the
impacts of the natural environment during the winter months on the
increasing number of bicyclists. Future research could focus on
determining what, if any, programmatic or built interventions could
ameliorate unfavorable cold- and wet-weather bicycling conditions.

\printbibliography
\end{document}